\documentclass[11pt,ngerman,a4paper]{article}
%Gummi|061|=)
\usepackage{amsmath}
\usepackage{a4wide}
\usepackage{url}
\usepackage{amsthm}
\usepackage{amsbsy}
\usepackage{amssymb}
\usepackage{inputenc}
\usepackage{rotating} 
\usepackage{here}
\usepackage{graphicx}
\usepackage{paralist}
\usepackage{selinput}
\usepackage[separate-uncertainty=true]{siunitx}
\usepackage{booktabs}
\sisetup{}
\SelectInputMappings{%
adieresis={ä},
germandbls={ß},
}
\title{\textbf{Versuch V703: Geiger-Müller-Zählrohr}}
\author{Martin Bieker\\
		Julian Surmann\\
		\\
		Durchgef\"{u}hrt am 27.05.2014\\
		TU Dortmund}
\date{}
\usepackage{graphicx}
\begin{document}
\renewcommand\tablename{Tabelle}
\renewcommand\figurename{Abbildung}
\maketitle
\thispagestyle{empty}
\newpage
\clearpage
\setcounter{page}{1}


\section{Einleitung}
Das Geiger-Müller-Zählrohr ist ein einfaches Messinstrument zur Messung der Intensität von ionisierender Strahlung. In diesem Versuch werden einige Kenndaten dieer Aperatur ermittelt.
\section{Theorie}
\subsection{Aufbau und Funktion}
\subsection{Totzeit und Nachtentladungen}
\subsection{Charektersitik}
\section{Durchführung}
\subsection{Aufnahme der Charakteristik}
\subsection{Oszillographische Messung der Totzeit}
\subsection{Bestimmung der Totzeit mit der Zwei-Quellen-Methode}
\subsection{Messung der pro Teilchen freigesetzten Ladungsmenge}
\section{Auswertung}
\begin{table}
\centering
\begin{tabular}{SSSSS}
\toprule
{U[V]} &{ N} &{ t[s]} &{ $I\left[\frac{1}{s}\right]$} &{ $\sigma_{I,rel}$[\%] }\\
\midrule
300.0 & 0.0 & 100.0 & 0 & nan\\
320.0 & 13617.0 & 250.0 & 54.5+-0.5 & 0.86\\
340.0 & 11192.0 & 200.0 & 56.0+-0.5 & 0.95\\
360.0 & 11231.0 & 200.0 & 56.2+-0.5 & 0.94\\
380.0 & 11601.0 & 200.0 & 58.0+-0.5 & 0.93\\
400.0 & 11410.0 & 200.0 & 57.0+-0.5 & 0.94\\
420.0 & 11459.0 & 200.0 & 57.3+-0.5 & 0.93\\
440.0 & 11496.0 & 200.0 & 57.5+-0.5 & 0.93\\
460.0 & 11433.0 & 200.0 & 57.2+-0.5 & 0.94\\
480.0 & 11379.0 & 200.0 & 56.9+-0.5 & 0.94\\
500.0 & 11457.0 & 200.0 & 57.3+-0.5 & 0.93\\
520.0 & 11437.0 & 200.0 & 57.2+-0.5 & 0.94\\
540.0 & 11376.0 & 200.0 & 56.9+-0.5 & 0.94\\
560.0 & 11564.0 & 200.0 & 57.8+-0.5 & 0.93\\
580.0 & 11620.0 & 200.0 & 58.1+-0.5 & 0.93\\
600.0 & 11333.0 & 200.0 & 56.7+-0.5 & 0.94\\
620.0 & 11382.0 & 200.0 & 56.9+-0.5 & 0.94\\
640.0 & 11449.0 & 200.0 & 57.2+-0.5 & 0.93\\
660.0 & 11414.0 & 200.0 & 57.1+-0.5 & 0.94\\
680.0 & 11507.0 & 200.0 & 57.5+-0.5 & 0.93\\
700.0 & 11642.0 & 200.0 & 58.2+-0.5 & 0.93\\
\bottomrule
\end{tabular}
\label{}
\caption{Messdaten und Fehlerangabe}
\end{table}


\begin{table}
\centering
\begin{tabular}{SSSS}
\toprule
{N} &{ t[s]} &{ $I\left[\frac{1}{s}\right]$} &{ $\sigma_{I,rel}$[\%] }\\
\midrule
17483.0 & 200.0 & 87.4+-0.7 & 0.76\\
20229.0 & 200.0 & 101.1+-0.7 & 0.70\\
13280.0 & 1000.0 & 13.28+-0.12 & 0.87\\
\bottomrule
\end{tabular}
\label{}
\caption{}
\end{table}

\section{Quellen}
\begin{enumerate}[{[}1{]}]
\item Entnommen der Praktikumsanleitung \textit{} der TU Dortmund. \\
Download am 01.06.14 unter:\\
 \url{http://129.217.224.2/HOMEPAGE/PHYSIKER/BACHELOR/AP/SKRIPT/V703.pdf}
\end{enumerate}

\section{Anhang}
\begin{itemize}
\item Tabellen
\item Auszug aus dem Messheft
\end{itemize}
\end{document}
